\documentclass{resumeclass}


\yourname{whh}
\yourage{18}
\yoursex{女}
\yourphonenumber{158xxxx8768}
\youremail{whhsurf@gmail.com}
\yourphoto{0.12}{photo.png} % 照片的大小和照片的文件路径


\begin{document}

% 个人信息
\resumehead


\cvsection{教育经历}
\cvevent{成都理工大学}{本科}{信息与计算科学}{2017.09—2021.06}
\begin{itemize}[topsep=0pt]
	\item \textbf{GPA}:4.0
	\item \textbf{主修课程}:数学分析,高等代数,概率论与数理统计,数值分析,复变函数,信号与系统
	\item \textbf{英语水平}:全国大学生英语等级考试CET-6:533分
	\item \textbf{毕业论文}:《我怀疑我是做了\LaTeX 的毕业设计》
\end{itemize}


\cvsection{荣誉证书}
\begin{itemize}[leftmargin=*]
	\item \cvaward{全国大学生数学建模竞赛}{一等奖}
	\item \cvaward{全国大学生数学竞赛}{二等奖}
	\item \cvaward{数学建模校内赛}{三等奖}
	\item \cvaward{MathorCup数学建模挑战赛}{一等奖}
	\item \cvaward{华为杯人工智能竞赛}{一等奖}
	\item \cvaward{计算机二级Python语言程序设计}{优秀}
\end{itemize}

\cvsection{技能专长}
\begin{itemize}[leftmargin=*]
	\item \textbf{Matlab}:熟练Matlab矩阵操作,在“xxx”比赛中利用matlab编写遗传算法解决了对目标优化问题。在数学建模比赛中,经常使用matlab的优化、拟合、神经网络等工具箱。
	\item \textbf{python}:较为熟悉numpy、pandas、matplotlib等科学计算模块,以及tensorflow、keras等深度学习框架。在人工智能竞赛中使用tensorflow完成了xxx识别。
	\item \textbf{\LaTeX}:掌握\LaTeX 的基本用法。能够编写\LaTeX 模板,这份简历是由本人使用\LaTeX 编写。
\end{itemize}



\cvsection{实践经历}
\cvevent{某某公司}{2018.07}{Java开发}{成都}
\begin{itemize}[leftmargin=*]
	\item 此系统为网上购物系统,用户可以用过该系统实现网上购物、注册、登录、浏览商
	品、订货、生成订单功能。也包括对普通商品增删改查功能和购物车内书的数量的增删改功能。
\end{itemize}


\end{document}